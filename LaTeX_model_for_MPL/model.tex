%A LaTeX model for SYSU General Phys. Lab. by Probfia Gao.
%用XeLaTeX编译
\documentclass[11pt,a4paper]{ctexart}

%在下面补全实验名,例如 实验BB3 光电效应实验。
\newcommand{\ExpeName}{实验D5 CCD原理与应用}

\usepackage{fancyhdr}
\usepackage{amsmath}
\usepackage{graphicx}
\usepackage[hmargin=1.25in,vmargin=1in]{geometry}
\usepackage{pdfpages}
\usepackage[colorlinks,
            linkcolor=red,
		 urlcolor=black]{hyperref}
\usepackage{cleveref}
\usepackage{float}


\crefname{equation}{}{}
\crefname{figure}{图}{图}
\crefname{footnote}{注释}{注释}
\crefname{table}{表}{表}

%\cpic{<尺寸>}{<文件名>}}用于生成居中的图片。
\newcommand{\cpic}[2]{
\begin{center}
\includegraphics[scale=#1]{#2}
\end{center}
}

%\cpicn{<尺寸>}{<文件名>}{<注释>}用于生成居中且带有注释的图片,其label为图片名。
\newcommand{\cpicn}[3]
{
\begin{figure}[H]
\cpic{#1}{#2}
\caption{#3\label{#2}}
\end{figure}
}

\newcommand{\beq}{\begin{equation}}
\newcommand{\eeq}{\end{equation}}
\newcommand{\bea}{\begin{equation}\begin{aligned}}
\newcommand{\eea}{\end{aligned}\end{equation}}

%输入单位和数学常数
%下面所有命令需在公式环境下使用
\newcommand{\e}{\mathrm{e}}   %自然常数e = \e
\newcommand{\im}{\mathrm{i}}   %虚数单位i = \im
\newcommand{\meter}{\mathrm{m}}      %单位/前缀 = \单位/前缀英文名
\newcommand{\newton}{\mathrm{N}}  
\newcommand{\joule}{\mathrm{J}}
\newcommand{\second}{\mathrm{s}}
\newcommand{\gram}{\mathrm{g}}
\newcommand{\ampere}{\mathrm{A}}
\newcommand{\kilo}{\mathrm{k}}
\newcommand{\milli}{\mathrm{m}}
\newcommand{\kelvin}{\mathrm{K}}
\newcommand{\mole}{\mathrm{mol}}
\newcommand{\volt}{\mathrm{V}}
\newcommand{\nano}{\mathrm{n}}
\newcommand{\degreeC}{^\circ \mathrm{C}}  %摄氏度符号 = \degreeC

\newcommand{\unit}[1]{\rm \ #1}
\newcommand{\emptyline}{\par \ \\ }

\pagestyle{fancy}

\fancyhead[L]{\footnotesize{中山大学物理与天文学院近代物理实验}}
\fancyhead[R]{\footnotesize{\ExpeName}}
\fancyfoot[C]{\thepage}

\begin{document}
%第一页
\cpic{0.255}{e1}%学生信息和计分表格
\begin{table}[H]
\centering
\begin{tabular}{|p{32mm}|p{32mm}|p{32mm}|p{32mm}|}
\hline
年级、专业: & 17级 物理学 & 组号: & 实验班6 \\ \hline
姓名: & 高寒 & 学号: & 17353019 \\ \hline
日期: & \today & 教师签名: &  \\ \hline
\end{tabular}
\end{table}
\begin{center}
\LARGE\textbf{{\ExpeName}}
\end{center}
\large{【实验报告注意事项】}
\begin{enumerate}
 \item 实验报告由三部分组成:
 \begin{enumerate}
  \item[1)]预习报告:(提前一周)认真研读\textbf{\uline{实验讲义}},弄清实验原理;实验所需的仪器设备、用具及其使用(强烈建议到实验室预习),完成讲义中的预习思考题;了解实验需要测量的物理量,并根据要求提前准备实验记录表格(由学生自己在实验前设计好,可以打印)。预习成绩低于50\%者不能做实验{\color{red} (实验D2和D3需要提前一周的周四完成预习报告交任课老师批改,批改通过后,才允许做实验)}。

  \item[2)]实验记录:认真、客观记录实验条件、实验过程中的现象以及数据。实验记录请用珠笔或者钢笔书写并签名({\color{red}用铅笔记录的被认为无效})。{\color{red}保持原始记录,包括写错删除部分,如因误记需要修改记录,必须按规范修改。}(不得输入电脑打印,但可扫描手记后打印扫描件);离开前请实验教师检查记录并签名。
  \item[3)]分析讨论:处理实验原始数据(学习仪器使用类型的实验除外),对数据的可靠性和合理性进行分析;按规范呈现数据和结果(图、表),包括数据、图表按顺序编号及其引用;分析物理现象(含回答实验思考题,写出问题思考过程,必要时按规范引用数据);最后得出结论。
 \end{enumerate}
 \textbf{实验报告}就是预习报告、实验记录、和数据处理与分析合起来,加上本页封面。
 \item 每次完成实验后的一周内交\textbf{实验报告}。
 \item 除实验记录外,实验报告其他部分建议双面打印。
\end{enumerate}
\ 
\\
\ 

\begin{flushright}                                                           %模板作者
\tiny{
A \LaTeX \ model for Modern Phys. Lab., SPA, SYSU by \em{\href{https://www.weibo.com/3532532974/profile?rightmod=1&wvr=6&mod=personinfo&is_all=1}{Probfia} Gao.}
}
\end{flushright}
\includepdf[pages=2-14]{preview.pdf}
\iffalse
\newpage%预习报告
\begin{center}
\LARGE{\textbf{\ExpeName}}
\end{center}
\textbf{【实验目的】}
\begin{enumerate}
 \item[1.] (目的1)
 \item[2.] (目的2)
\end{enumerate}
\textbf{【仪器用具】}
%将讲义中的表格截图保存为t1在该文件夹下后删去下一行之前的%符号,合理调整scale参数。
%cpic{0.3}{t1}
%或者自己去 https://www.tablesgenerator.com/ 做一个表。
\textbf{【原理概述】}\par
该实验\\
\textbf{【实验前思考题】}
\begin{enumerate}
 \item[1.] (问题1)
 \item[2.] (问题2)
\end{enumerate}

\newpage%实验记录
%\cpic{0.255}{e2}%学生信息表格
\begin{table}[H]
\centering
\begin{tabular}{|p{32mm}|p{32mm}|p{32mm}|p{32mm}|}
\hline
年级、专业: & 17级 物理学 & 组号: & 实验班6 \\ \hline
姓名: & 高寒 & 学号: & 17353019 \\ \hline
日期: &  & 实验地点: & 珠海教学楼 A102 \\ \hline
学生签名: &  \includegraphics[scale=0.09]{sign}& 评分: &  \\ \hline
日期: &  & 教师签名: &  \\ \hline
\end{tabular}
\end{table}

\begin{center}
\LARGE{\textbf{\ExpeName}}
\end{center}
\textbf{【实验内容、步骤、结果】}
\par
(表格)
\newline
\textbf{【实验过程中遇到问题记录】}\par
\cpic{0.81}{none}
\fi
%生成最终报告时将上面内容全部删除或注释(用\iffalse \fi),将扫描得到的实验报告保存为Record.pdf在LaTeX model for GPL下,将下行命令的注释号删去。注意根据实际页数调整pages参数。
\includepdf[pages=1-9]{Record}


\newpage%分析与讨论
%\cpic{0.255}{e3}%学生信息表格
\begin{table}[H]
\centering
\begin{tabular}{|p{32mm}|p{32mm}|p{32mm}|p{32mm}|}
\hline
年级、专业: & 17级 物理学 & 组号: & 实验班6 \\ \hline
姓名: & 高寒 & 学号: & 17353019 \\ \hline
日期: & \today &  &  \\ \hline
评分: &  & 教师签名: &  \\ \hline
\end{tabular}
\end{table}
\begin{center}
\LARGE\textbf{{\ExpeName}}
\end{center}
\textbf{【分析与讨论】}\\


\bibliographystyle{IEEEtran}
\bibliography{model.bib}

\end{document}